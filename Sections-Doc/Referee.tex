This paper contributes to the literature studying the effect of creating artificial borderlines on long-term economic performance. Specifically, this paper examines how imposing common jurisdiction on a group of communities with common culture/ethnicity affects the economic income in the long run. To do so, the paper uses the Reservations of Native Americans as a quasi-experimental set-up to analyze the effects of initial difference in forced coexistence between different communities on current income. The paper explains how in the process of forming Reservations, the US government avoided integrating more than one tribe within a single Reservation. However, tribes had different political and social structures ex-ante which provides a source of variation in terms of initial governance even while having language, culture, and other traits that were common to the tribe. 

The paper uses two different strategies to estimate the effect of forced coexistence. First, it runs an OLS regression assuming that forced coexistence is random conditional on random observable characteristics and tribe fix effects. Nevertheless, the paper acknowledges that integrating several bands from the same tribe when forming a reservation could have had a decision rule based on unobserved characteristics of the tribe. Consequently, the paper addresses this concern by using an IV estimation based on mining rushes. The author argues that mining rushes changed the incentives of the government to create formation while being orthogonal to tribe unobserved characteristics. The author estimates that forced coexisting led to a decrease of approximately 30\% of current income. These estimates are robust to the inclusion of covariates and fix effects. Additionally, the results of the IV exercise are still significant and close to the OLS estimates. 

Finally, the paper goes one step further by trying to unveil the mechanisms that drive the current differences in income per capita. The author argues that forcing cohabitation between bands implied high political conflict which at the same time yielded a more uncertain business environment. 

\textbf{Comments.} Overall the paper does a good job of presenting a compelling argument. The main point is convincing and it lingers after a careful reading. Moreover, it pushes the literature by going further from the low-hanging fruit mechanisms based on culture. It finds a creative setting for a quasi-natural experiment and digs deep to unveil the economic mechanisms that are impacting the long-run economic performance. In addition, the creation of the data set shows comprehensive work of archives.

On the flip side, the paper has some potential areas to improve. First and foremost, the author estimates, even while being significant seem too high and it is a hard sell. 30\% divergence in income seems like a lot considering that the author calculated the effects of coexistence while holding fixed cultural factors which are typically the drivers of the first-order effects. More importantly, there is no discussion about the magnitude of the estimates and what mechanism can make them believable. 

Second, the IV exercise has a couple of loose ends that would require a more detailed discussion. The argument for instrument exogeneity, which is probably one of the more important discussions is limited to no more than three sentences. It comes up as weak and raises more questions than what it solves. Similarly, there is no discussion about the monotonicity of the instrument. There can potentially be offsetting effects without monotonicity that would make the estimates even bigger and a tougher sale. This discussion becomes even more relevant in the estimation that uses two different instruments. Finally, the discussion on what the IV is identifying could be expanded. Specifically, even though the author acknowledges the limitation of estimating the LATE and identifies the group of compliers, it fails to explain why is this the relevance of estimating the effect solely on the group of compliers. In addition to discussing why the OLS estimates and IV estimates differ in magnitude, a discussion on the relevance of estimating the LATE within the group  of compliers should be included. 

Lastly, the paper tests mechanisms using reports from newspapers. Though this seems like a creative idea, it lacks a discussion of the potential biases of using this source of information. 

On a more technical note, there are issues with the empirical strategy that raises concerns about the empirical exercises. First, in all of the specifications, the author includes controls that are determined after the treatment. For instance, having a casino, population and unemployment are just a few examples of controls that are determined after treatment and can be a source of bias to the regression estimates.  

Also, the inclusion of state fix effects in the IV regressions poses a threat to the result as over-fitting the first stage will not solve the endogeneity problem. Note that the $R^2$ is over 0.5 but moreover, when comparing the mean of the predicted values of the first stage to the actual mean of the endogenous variable, I found them to be identical. Furthermore, the F-statistic for the weak instrument seems low enough to be worthy of a discussion and reporting the confidence intervals. Last but not least, not that the coefficient of the OLS regression decrease as more controls are added whereas on the IV identification the coefficient increases as more controls are added. Even though the IV and OLS are identifying different parameters, the trend described and the fact that both coefficients are sufficiently close deserves a deeper analysis. 
